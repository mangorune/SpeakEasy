\documentclass[10pt,letter]{article}

% Multilingual Christmas Greeting Card

% This greeting card was created for SpeakEasy as a part of "German in the
% Working World", taken Fall quarter 2014 at Portland State University.

% ------------------------------------------------------------------------------

% AUTHORSHIP, COPYRIGHT and LICENSE

% Written in 2014 by Benjamin Johnson.

% To the extent possible under law, the author(s) have dedicated all copyright
% and related and neighboring rights to this source code to the public domain
% worldwide. This source code is distributed without any warranty.

% You should have received a copy of the CC0 Public Domain Dedication along with
% this source code. If not, see
% <http://creativecommons.org/publicdomain/zero/1.0/>.

% ------------------------------------------------------------------------------

\usepackage{./speakeasy_greeting_card}

\hypersetup{
  colorlinks={true},
  linkcolor={black},
  urlcolor={Black},
  pdfauthor={Benjamin Johnson},
  pdfcopyright={To the extent possible under law, the author(s) have waived
    all copyright and related or neighboring rights to this work.},
  pdflicenseurl={https://creativecommons.org/publicdomain/zero/1.0/}
}

% ------------------------------------------------------------------------------

\usepackage{polyglossia}

\setdefaultlanguage{english}

\setotherlanguage{spanish}
\setotherlanguage{portuges}
\setotherlanguage{german}
\setotherlanguage{french}
\setotherlanguage{vietnamese}
\setotherlanguage{italian}

% Filipino / Tagalog not supported by Polyglossia. Using generic fontspec macros

\setotherlanguage{russian}

% CJK not supported by Polyglossia.
\usepackage{xeCJK}

\setotherlanguage{arabic}

%% Typeface choices

\defaultfontfeatures{Ligatures={Common,TeX}}

% Used by any language without another font defined
\setmainfont{Noto Sans}

% English
% Used as default language for the document
\newfontfamily\englishfont[Script=Latin]{Noto Sans}

% Latin scripts, non-English

\newfontfamily\spanishfont[Script=Latin]{Noto Sans}
\newfontfamily\portuguesefont[Script=Latin]{Noto Sans}
\newfontfamily\germanfont[Script=Latin]{Noto Sans}
\newfontfamily\frenchfont[Script=Latin]{Noto Sans}
\newfontfamily\vietnamesefont[Script=Latin]{Noto Sans}
\newfontfamily\italianfont[Script=Latin]{Noto Sans}

\newfontfamily\tagalogfont[Script=Latin]{Noto Sans}

% Alphabetic scripts, non-Latin
\newfontfamily\russianfont[Script=Cyrillic]{Noto Sans}

% Left-to-Right scripts, non-alphabetic
% (All these are Logographic / Syllabic)

% These modern open source type families are not working due to bugs somewhere
% in the *TeX pipeline.
% \setCJKfamilyfont{zhrm}{Noto Sans S Chinese}
% \setCJKfamilyfont{jarm}{Noto Sans Japanese}
% \setCJKfamilyfont{korm}{Noto Sans Korean}
% \setCJKfamilyfont{zhrm}{Source Han Sans CN}
% \setCJKfamilyfont{jarm}{Source Han Sans JP}
% \setCJKfamilyfont{korm}{Source Han Sans KR}

\setCJKfamilyfont{zhrm}{DFKai-SB}
\setCJKfamilyfont{jarm}{Yu Mincho}
\setCJKfamilyfont{korm}{UnBatang}

\newcommand\chinesefont{\CJKfamily{zhrm}\CJKnospace}
\newcommand\japanesefont{\CJKfamily{jarm}\CJKnospace}
\newcommand\koreanfont{\CJKfamily{korm}\CJKspace}

% Right-to-Left scripts

\newfontfamily\arabicfont[Script=Arabic]{Noto Naskh Arabic}

% ------------------------------------------------------------------------------

%% Language native names

\newcommand{\langEnglish}{English}

\newcommand{\langSpanish}{Español}
\newcommand{\langGerman}{Deutsch}
\newcommand{\langVietnamese}{Tiếng Việt}
\newcommand{\langRussian}{Русский}
\newcommand{\langFrench}{Français}

\newcommand{\langJapanese}{日本語}
\newcommand{\langKorean}{한국어}
\newcommand{\langTagalog}{Tagalog}
\newcommand{\langArabic}{العربية}
\newcommand{\langItalian}{Italiano}

%% Multilingual phrase definitions

\newcommand{\mcEnglish}{Merry Christmas}

\newcommand{\mcSpanish}{Feliz Navidad}
\newcommand{\mcGerman}{Fröhliche Weihnachten} % alt: {Frohe Weihnachten}
\newcommand{\mcVietnamese}{Chúc Giáng Sinh Vui Vẻ} % alt: {Giáng Sinh Vui Vẻ}
\newcommand{\mcRussian}{Счастливого Рождества} % alt: {Весёлого Рождества}
\newcommand{\mcFrench}{Joyeux Noël}

\newcommand{\mcJapanese}{メリークリスマス}
\newcommand{\mcKorean}{메리 크리스마스} % alt: {즐거운 성탄절}
\newcommand{\mcTagalog}{Malagayang Pasko}
\newcommand{\mcArabic}{{مجيد} {ميلاد} {عيد}} % these are reversed for Reasons
\newcommand{\mcItalian}{Buon Natale}

% ------------------------------------------------------------------------------

\begin{document}

% ------------------------------------------------------------------------------

\begin{frontcover}
  % All on separate lines, for easy viewing
  % \textenglish{\mcEnglish{}}

  % \textspanish{\mcSpanish{}}

  % \textgerman{\mcGerman{}}

  % \textvietnamese{\mcVietnamese{}}

  % \textrussian{\mcRussian{}}

  % \textfrench{\mcFrench{}}

  % {\japanesefont{}\mcJapanese{}}

  % {\koreanfont{}\mcKorean{}}

  % {\tagalogfont{}\mcTagalog{}}

  % \textarabic{\mcArabic{}}

  % \textitalian{\mcItalian{}}

  %\vfill

  % An alternate style of flake I was playing with
  % \begin{tikzpicture}[l-system={step=1pt, order=7, angle=10}]
  %   \pgfdeclarelindenmayersystem{flake}{
  %     \symbol{M}{\pgflsystemdrawforward}
  %     \symbol{R}{\pgflsystemdrawforward}
  %     \symbol{L}{\pgflsystemdrawforward}
  %     \rule{C -> HHHHHH}
  %     \rule{6 -> ++++++}
  %     \rule{H -> [B]6}
  %     \rule{B -> M}
  %     \rule{M -> M[------L][++++++R]M[------L][++++++R]M}
  %     \rule{R -> R[++++R]M[++++R]-MM}
  %     \rule{L -> L[----L]M[----L]+MM}
  %   }
  %   \draw l-system [l-system={flake, axiom=C}];
  % \end{tikzpicture}

  \begin{center}
    \begin{tikzpicture}[l-system={step=1pt, order=5, angle=10}]
      \pgfdeclarelindenmayersystem{flake}{
        \symbol{M}{\pgflsystemdrawforward}
        \symbol{R}{\pgflsystemdrawforward}
        \symbol{L}{\pgflsystemdrawforward}
        \rule{H -> [M]++++++}
        \rule{M -> M[------L][++++++R]M[------R][++++++L]M}
        \rule{R -> M-M-M-M}
        \rule{L -> M+M+M+M}
      }
      \draw l-system [l-system={flake, axiom=HHHHHH}];

      % It would be nice to make this more concise and clear,
      % but I'm short on time.
      \path
      [postaction={decorate, decoration={text format delimiters={|}{|},
          text along path, text align/align=center,
          % text align/fit to path stretching spaces=true,
          raise=-0.5\baselineskip, reverse path=true,
          text align/left indent={1.25663706\panelwidth}, % 3/3 \pi * radius
          text align/right indent={0.83775804\panelwidth}, % 2/3 \pi * radius
          text={|\frenchfont|  \mcFrench{}}
        }}]
      [postaction={decorate, decoration={text format delimiters={|}{|},
          text along path, text align/align=center,
          % text align/fit to path stretching spaces=true,
          raise=-0.5\baselineskip, reverse path=true,
          text align/left indent={1.67551608\panelwidth}, % 4/3 \pi * radius
          text align/right indent={0.41887902\panelwidth}, % 1/3 \pi * radius
          text={|\koreanfont|  \mcKorean{}}
        }}]
      [postaction={decorate, decoration={text format delimiters={|}{|},
          text along path, text align/align=center,
          % text align/fit to path stretching spaces=true,
          raise=-0.5\baselineskip, reverse path=true,
          text align/left indent={2.0943951\panelwidth}, % 5/3 \pi * radius
          text align/right indent={0\panelwidth}, % 0/3 \pi * radius
          text={|\arabicfont|  \mcArabic{}}
        }}]
      [postaction={decorate, decoration={text format delimiters={|}{|},
          text along path, text align/align=center,
          % text align/fit to path stretching spaces=true,
          raise=-0\baselineskip, reverse path=false,
          text align/left indent={1.25663706\panelwidth}, % 3/3 \pi * radius
          text align/right indent={0.83775804\panelwidth}, % 2/3 \pi * radius
          text={|\spanishfont|  \mcSpanish{}}
        }}]
      [postaction={decorate, decoration={text format delimiters={|}{|},
          text along path, text align/align=center,
          % text align/fit to path stretching spaces=true,
          raise=-0\baselineskip, reverse path=false,
          text align/left indent={1.67551608\panelwidth}, % 4/3 \pi * radius
          text align/right indent={0.41887902\panelwidth}, % 1/3 \pi * radius
          text={|\tagalogfont|  \mcTagalog{}}
        }}]
      [postaction={decorate, decoration={text format delimiters={|}{|},
          text along path, text align/align=center,
          % text align/fit to path stretching spaces=true,
          raise=-0\baselineskip, reverse path=false,
          text align/left indent={2.0943951\panelwidth}, % 5/3 \pi * radius
          text align/right indent={0\panelwidth}, % 0/3 \pi * radius
          text={|\italianfont|  \mcItalian{}}
        }}]
      circle[radius=0.4\panelwidth];

      \path
      [postaction={decorate, decoration={text format delimiters={|}{|},
          text along path, text align/align=center,
          % text align/fit to path stretching spaces=true,
          raise=-0.5\baselineskip, reverse path=true,
          text align/left indent={1.413716693\panelwidth}, % 2/2 \pi * radius
          text align/right indent={0.706858346\panelwidth}, % 1/2 \pi * radius
          text={|\germanfont|  \mcGerman{}}
        }}]
      [postaction={decorate, decoration={text format delimiters={|}{|},
          text along path, text align/align=center,
          % text align/fit to path stretching spaces=true,
          raise=-0.5\baselineskip, reverse path=true,
          text align/left indent={2.120575039\panelwidth}, % 3/2 \pi * radius
          text align/right indent={0\panelwidth}, % 0/2 \pi * radius
          text={|\vietnamesefont|  \mcVietnamese{}}
        }}]
      [postaction={decorate, decoration={text format delimiters={|}{|},
          text along path, text align/align=center,
          % text align/fit to path stretching spaces=true,
          raise=-0\baselineskip, reverse path=false,
          text align/left indent={1.413716693\panelwidth}, % 2/2 \pi * radius
          text align/right indent={0.706858346\panelwidth}, % 1/2 \pi * radius
          text={|\japanesefont|  \mcJapanese{}}
        }}]
      [postaction={decorate, decoration={text format delimiters={|}{|},
          text along path, text align/align=center,
          % text align/fit to path stretching spaces=true,
          raise=-0\baselineskip, reverse path=false,
          text align/left indent={2.120575039\panelwidth}, % 3/2 \pi * radius
          text align/right indent={0\panelwidth}, % 0/2 \pi * radius
          text={|\englishfont|  \mcEnglish{}}
        }}]
      circle[radius=0.45\panelwidth] ;

      \path
      [postaction={decorate, decoration={text format delimiters={|}{|},
          text along path, text align/align=center,
          % text align/fit to path stretching spaces=true,
          raise=-0\baselineskip, reverse path=false,
          text align/left indent={1.570796325\panelwidth}, % \pi * radius
          text align/right indent={0.0cm},
          text={|\russianfont|  \mcRussian{}}
        }}]
      circle[radius=0.5\panelwidth] ;
    \end{tikzpicture}
  \end{center}

  %\vfill
\end{frontcover}

% ------------------------------------------------------------------------------

\begin{backcover}

  \vfill

  % (a) SpeakEasy doesn't have a consistent logo;
  % (b) The logos for SpeakEasy and Portland State University are not in the
  % public domain;
  % Therefore those logos are not included.
  \begin{center}
    \small
    SpeakEasy
  \end{center}

  \vfill

  \footnotesize

  % SpeakEasy brief and link
  \begin{wrapfigure}[4]{r}{4\baselineskip}
    \centering
    \vspace{-1.4\baselineskip}
    % http://speakeasy.wll.pdx.edu
    \href{https://goo.gl/uoKYLW}{
      \XeTeXLinkBox{
        \qrcode[height=4\baselineskip]{https://goo.gl/uoKYLW}
      }
    }
  \end{wrapfigure}
  Designed on behalf of SpeakEasy, a student
  entrepreneurship project within the Department of World Languages and
  Literatures at Portland State University.
  \hfill \href{http://speakeasy.wll.pdx.edu/}{speakeasy.wll.pdx.edu}
  \\[0.5 em]

  % Language Design
  SpeakEasy products are a celebration of language. This card includes the most
  widely spoken languages of residents in Portland, Oregon:
  Arabic (\textarabic{\langArabic{}}),
  English, % native language
  French (\textfrench{\langFrench{}}),
  German (\textgerman{\langGerman{}}),
  Italian (\textitalian{\langItalian{}}),
  Japanese ({\japanesefont{}\langJapanese{}}),
  Korean ({\koreanfont{}\langKorean{}}),
  Russian (\textrussian{\langRussian{}}),
  Spanish (\textspanish{\langSpanish{}}),
  Tagalog, % same in native language
  Vietnamese (\textvietnamese{\langVietnamese{}}).
  \\[0.5 em]

  % Sustainability, Mr Ellie Pooh, and link to Mr. Ellie Pooh
  % Modify this to suit the instantiated product!
  \begin{wrapfigure}[4]{r}{4\baselineskip}
    \centering
    \vspace{-1.4\baselineskip}
    % http://mrelliepooh.com/
    \href{https://goo.gl/NXG25H}{
      \XeTeXLinkBox{
        \qrcode[height=4\baselineskip]{https://goo.gl/NXG25H}
      }
    }
  \end{wrapfigure}
  Printed on 100\% recycled paper by Mr. Ellie Pooh, at least 50\% recycled
  elephant dung. This program supports conservation and human-elephant conflict
  resolution in Sri Lanka. \hfill
  \href{http://mrelliepooh.com/}{mrelliepooh.com}

  \vfill

  % About author; link to GitHub Repository with source and pdf
  \begin{wrapfigure}[4]{r}{4\baselineskip}
    \centering
    \vspace{-1.4\baselineskip}
    % https://github.com/mangorune/SpeakEasy/
    \href{https://goo.gl/B62dF0}{
      \XeTeXLinkBox{
        \qrcode[height=4\baselineskip]{https://goo.gl/B62dF0}
      }
    }
  \end{wrapfigure}
  Author: Benjamin Johnson (mangorune)

  \LaTeX{} source and PDF files % for mangorune's SpeakEasy products are
  available on GitHub. % Modify to your needs.

  \hfill
  \href{https://github.com/mangorune/SpeakEasy/}{github.com/mangorune/SpeakEasy}

  % Public Domain Dedication (CC0) with link to legal code.
  \begin{center}
    \resizebox{\panelwidth}{!}{\ccZeroNotice}
  \end{center}
\end{backcover}

% ------------------------------------------------------------------------------

\begin{insideleft}
  % Maybe leave this entirely blank on real cards
  % This is the insideleft panel. \\

  % \setmainfont{Noto Sans} ``Noto Sans'' \\
  % Lorem ipsum dolor sit amet, consectetur adipiscing elit, sed do eiusmod tempor
  % incididunt ut labore et dolore magna aliqua. Ut enim ad minim veniam, quis
  % nostrud exercitation ullamco laboris nisi ut aliquip ex ea commodo
  % consequat. \\

  % \setmainfont{Source Sans Pro} ``Source Sans Pro'' \\
  % Lorem ipsum dolor sit amet, consectetur adipiscing elit, sed do eiusmod tempor
  % incididunt ut labore et dolore magna aliqua. Ut enim ad minim veniam, quis
  % nostrud exercitation ullamco laboris nisi ut aliquip ex ea commodo
  % consequat. \\

  % \setmainfont{Open Sans} ``Open Sans'' \\
  % Lorem ipsum dolor sit amet, consectetur adipiscing elit, sed do eiusmod tempor
  % incididunt ut labore et dolore magna aliqua. Ut enim ad minim veniam, quis
  % nostrud exercitation ullamco laboris nisi ut aliquip ex ea commodo
  % consequat. \\

  % \vfill

  % bottom of panel
\end{insideleft}

% ------------------------------------------------------------------------------

\begin{insideright}
  % Maybe leave this entirely blank on real cards
  % This is the insideright panel. \\

  % \setmainfont{Noto Serif} ``Noto Serif'' \\
  % Lorem ipsum dolor sit amet, consectetur adipiscing elit, sed do eiusmod tempor
  % incididunt ut labore et dolore magna aliqua. Ut enim ad minim veniam, quis
  % nostrud exercitation ullamco laboris nisi ut aliquip ex ea commodo
  % consequat. \\

  % \setmainfont{Source Serif Pro} ``Source Serif Pro'' \\
  % Lorem ipsum dolor sit amet, consectetur adipiscing elit, sed do eiusmod tempor
  % incididunt ut labore et dolore magna aliqua. Ut enim ad minim veniam, quis
  % nostrud exercitation ullamco laboris nisi ut aliquip ex ea commodo
  % consequat. \\

  % \setmainfont{EB Garamond} ``EB Garamond'' \\
  % Lorem ipsum dolor sit amet, consectetur adipiscing elit, sed do eiusmod tempor
  % incididunt ut labore et dolore magna aliqua. Ut enim ad minim veniam, quis
  % nostrud exercitation ullamco laboris nisi ut aliquip ex ea commodo
  % consequat. \\

  % \vfill

  % bottom of panel
\end{insideright}

% ------------------------------------------------------------------------------

\end{document}

% ------------------------------------------------------------------------------
